\chapter{Riscos associados ao Produto}
\hspace{5mm} Neste capítulo aborda-se os riscos associados à aplicação. Desta forma, existem vários riscos de segurança para os viajantes. Na verdade, nunca se sabe as pessoas que podem aparecer para a partilha do carro, pois podem ter segundas intenções, por exemplo, com o intuito de assaltar pessoas. No entanto, haverá um investimento necessário de fiscalização aos viajantes da aplicação, e obviamente a recomendação de viajarem apenas com pessoas com boa classificação ou conhecidos.

\hspace{5mm} Outros riscos possíveis relacionam-se com condução dos motoristas, que afetam todos os passageiros do veículo.

\hspace{5mm} Por outro lado, acidentes não provocados pelos motoristas, mas por questões externas, impossíveis de evitar, e que afetam todos os viajantes do veículo.

\hspace{5mm} A nível de segurança dos dados, tudo será feito para se evitar exposição dos dados indesejados, com equipas qualificadas para garantir a segurança dos mesmos, no entanto, existe sempre esse risco.

\hspace{5mm} A aplicação não consegue garantir que as pessoas vão cumprir as deslocações. No entanto, em caso de reportação dessas situações, o causador das mesmas, em caso de ausência de uma boa explicação, será desclassificado ou banido da aplicação.

\hspace{5mm}Por fim, e não menos importante, a inserção de documentação falsa (Cartão de cidadão ou seguro do carro) poderá a acontecer, no entanto, terá que existir uma equipa e mecanismos de fiscalização destes documentos.



