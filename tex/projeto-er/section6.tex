\chapter{O âmbito do Trabalho}

\hspace{5mm} Este capítulo, refere os limites do domínio do nosso sistema por isso define o ambiente em que o mesmo se encontra. Definindo o ambiente de forma mais clara é possível detalhar com mais precisão e menor ambiguidade o domínio do sistema e ter melhor noção do impacto do novo sistema no mercado atual.

\section{A Situação Atual}

\hspace{5mm} A introdução de um novo sistema \textbf{implica sempre alterações}, caso contrário nem faria sentido o desenvolvimento e implementação desse sistema, o objetivo deve ser melhorar ou criar algo novo.

\hspace{5mm} Desta forma deve ser feito um estudo da situação atual das entidades que serão remodeladas com o objetivo de verificar como deverá ser efetuada essa alteração da forma mais eficiente.

\hspace{5mm} A própria empresa que desenvolveu a aplicação terá de realizar a \textbf{manutenção do sistema}, dessa forma a primeira alteração é a introdução de uma equipa encarregue desse trabalho.

\hspace{5mm} Visto que o novo sistema exige, por questões de segurança, que os utilizadores forneçam o seu id civil, é necessário que o cliente do sistema tenha uma equipa de recursos humanos a \textbf{fiscalizar} este documento e a sua validade.

\hspace{5mm} Do mesmo modo o seguro do veículo do motorista também necessita de ser válido e por isso fiscalizado.

\hspace{5mm} Caso o sistema tenha alta adesão por parte dos seus consumidores é possível que se verifique um diminuição de trânsito e consequente diminuição de emissões de gases prejudiciais para a atmosfera, contribuindo para uma sociedade mais sustentável.

\section{O Contexto do Trabalho}

\hspace{5mm} De forma a clarificar o domínio do sistema deve ser estudado o ambiente no qual o mesmo vai ser implementado.

\hspace{5mm} Inicialmente convém identificar \textbf{sistemas semelhantes}, neste momento o mercado contém algumas aplicações que fornecem serviços semelhantes sendo as principais: \textcolor{blue}{\href{https://www.uber.com}{uber}}, \textcolor{blue}{\href{https://bolt.eu}{bolt}} e \textcolor{blue}{\href{https://www.blablacar.pt/}{blablacar}}. Assim a equipa estudou estas aplicações de forma a identificar funcionalidades que as mesmas \textbf{não têm ou podem ser melhoradas} introduzindo-as no novo sistema a desenvolver. Outra vantagem de estudar estas aplicações é o conhecimento do estado atual da futura concorrência no mercado.

\hspace{5mm} Desta forma uma funcionalidade que foi identificada que as aplicações já existentes \textbf{não fornecem} é a possibilidade dos viajantes poderem \textbf{estabelecer o preço da deslocação entre eles sem necessitarem de efetuar transações monetárias} pela própria aplicação. Visto que é uma funcionalidade possivelmente útil e única, é uma vantagem a mesma pertencer ao novo sistema. 

\hspace{5mm} Outra funcionalidade que o sistema oferece é o \textbf{cálculo automático da estimativa do valor da deslocação}, de forma a verificar a possibilidade de implementação desta funcionalidade a equipa informou-se da existência de serviços que já realizam esta operação.

\hspace{5mm} Desta forma um serviço já existente é a api \textcolor{blue}{\href{https://api.viamichelin.com}{via-michelin}} que de acordo com as informações do carro: marca, modelo, motor, ano e tipo de combustível; o ponto de origem e ponto de destino da deslocação calcula com precisão a estimativa do custo. Note-se que apenas identificamos um possível serviço a utilizar não significa que no momento de implementação da solução, seja a melhor decisão a tomar, visto que podem existir desvantagens tais como o custo de utilização da api ou dependência de serviços externos ao sistema.







