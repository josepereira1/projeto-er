\chapter{Convenções de nomenclatura e definições}

\section{Convenções de nomenclatura}
\begin{itemize}
    \item Deslocação
    \item Utilizador
    \item Viajante
    \item Administrador
    \item Perfil
    \item Classificação
    \item Comentários
    \item Veículo
    \item Chat 
\end{itemize}{}

\section{Políticas de negócio}
\hspace{5mm} Na fase inicial, o objetivo principal da aplicação centra-se em reunir pessoas, com a finalidade de efetuar uma deslocação, de forma a aproveitar os recursos automóveis, reduzindo os custos de deslocação para todos os intervenientes, bem como o trânsito. 

\section{Definições}
\begin{itemize}
    \item \textbf{Deslocação:} reunião entre utilizadores previamente planeada, organizada e coordenada de forma a contemplar o maior número de pessoas em um mesmo espaço virtual para efetuarem uma deslocação, podendo esta viagem ser frequente ou não.
    \item \textbf{Utilizador:} ator que utiliza a aplicação. O utilizador pode ser o viajante ou o administrador da aplicação.
    \item \textbf{Viajante:} são os utilizadores que efetuam as deslocações, existindo dois tipos:
    \begin{itemize}
        \item motorista
        \item passageiro
    \end{itemize}
    \item \textbf{Administrador:} utilizador da aplicação que monitoriza a mesma, através de estatísticas.
    \item \textbf{Perfil:} conjunto de informações sobre um viajante da aplicação, tais como: classificação, comentários sobre o mesmo, suas deslocações, entre outras informações.
    \item \textbf{Classifição:} está presente no perfil de cada viajante, servindo para avaliar os intervenientes numa deslocação. No fim de cada deslocação, atribui-se a cada viajante,uma classificação, sobre o seu comportamento na mesma. 
    \item \textbf{Comentários:} do mesmo modo, que a classificação, também estão associados a cada utilizador comentários sobre o comportamento do mesmo na aplicação. Os comentários são feitos no fim de cada deslocação.
    \item \textbf{Veículo:} meio de transporte que um motorista possuí para efetuar as deslocações nas quais está inserido.
    \item \textbf{Chat:} forma de comunicação em tempo real dos intervenientes de uma deslocação.
\end{itemize}{}