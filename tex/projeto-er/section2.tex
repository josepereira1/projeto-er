\chapter{Partes Interessadas}
\section{Cliente}\label{0:0.2.1}
\hspace{5mm} O \emph{cliente} deste projeto é a administração da empresa \textbf{X}, também responsável pelo desenvolvimento do mesmo.

\section{Consumidor}\label{0:0.2.2}
\hspace{5mm} Os consumidores da aplicação \textbf{ShareCar}, serão todas as pessoas que se deslocam entre dois pontos, onde seja necessária a utilização de um automóvel e que estejam disponíveis para partilhar o mesmo com outros utilizadores. Exemplos de pessoas para consumidores serão: universitários, pessoas que trabalham no mesmo lugar ou próximo, pessoas que façam viagens longas como por exemplo de Braga a Lisboa, etc.

\section{Consumidores de outros sistemas}\label{0:0.2.3}

\hspace{5mm} Uma vez já existem no mercado atual produtos que têm um objetivo semelhante ao do nosso é importante analisar a concorrência e contabilizar os consumidores desses sistemas como possíveis partes interessadas.

\hspace{5mm} Assim identificamos três sistemas a ter em conta: \href{https://www.uber.com}{Uber}, \href{https://bolt.eu/en/}{Bolt} e \href{https://www.blablacar.pt/}{BlaBlaCar}. 


\section{Outras partes interessadas}
\begin{itemize}
    \item Cliente (\ref{0:0.2.1})
    \item Consumidor (\ref{0:0.2.2})
    \item Consumidores de outros sistemas (\ref{0:0.2.3})
    \item Ministério do Ambiente
    \item Ministério das Finanças
    \item Orgãos regulamentadores
    \item Colaboradores da empresa X
    \item Investidores da empresa X
    \item Ambientalistas
    \item Organizações não governamentais
    \item Engenheiros de Software
    \item Programadores
\end{itemize}

\section{Utilizadores do Produto}
\hspace{5mm} Os utilizadores do produto são aqueles interessados em utilizar a aplicação para encontrar pessoas, a fim de dividir despesas e partilhar um automóvel ao se deslocarem entre dois ou mais pontos. Destes utilizadores, pode-se destacar os seguintes:

\begin{itemize}
        \item Professores de Universidades, escolas Básicas e Secundárias
        \begin{itemize}
            \item Professores de uma mesma escola que residam proximamente ou no caminho para a escola em questão, podem partilhar um automóvel.
            \item O papel deste utilizador será de motorista ou de passageiro.
            \item Os conhecimentos tecnológicos necessários serão a capacidade de abrir e utilizar a aplicação \textbf{ShareCar} em um smartphone, Desktop ou Laptop.
        \end{itemize}{}
        \item Estudantes Universitários
        \begin{itemize}
            \item Estudantes da mesma universidade ou de universidades locais e que residam proximamente, podem partilhar o automóvel.
            \item O papel deste utilizador será de motorista ou de passageiro.
            \item Os conhecimentos tecnológicos necessários serão a capacidade de abrir e utilizar a aplicação \textbf{ShareCar} em um smartphone, Desktop ou Laptop.
        \end{itemize}{}
        \item Pessoas da mesma zona residêncial que trabalham no mesmo lugar ou próximo
        \begin{itemize}
            \item O papel deste utilizador será de motorista ou de passageiro.
            \item Os conhecimentos tecnológicos necessários serão a capacidade de abrir e utilizar a aplicação \textbf{ShareCar} em um smartphone, Desktop ou Laptop.
        \end{itemize}{}
        \item Pessoas que costumam fazer grandes viagens frequentemente (p.e. Braga - Lisboa)\begin{itemize}
            \item O papel deste utilizador será de motorista ou de passageiro.
            \item Os conhecimentos tecnológicos necessários serão a capacidade de abrir e utilizar a aplicação \textbf{ShareCar} em um smartphone, Desktop ou Laptop.
        \end{itemize}{}
\end{itemize}{}

\section{Personas}
\hspace{5mm} De seguida apresentam-se três Personas possíveis da aplicação \textbf{ShareCar}.

\subsection{Paulo Monteiro}
\begin{itemize}
    \item Idade: 20 anos
    \item Estado Civil: Solteiro
    \item Habilitações: 12º Ano
    \item Profissão: Estudante
    \item Residência: Guimarães
    \item \textbf{Life style:} O Paulo Monteiro gosta de sair com os amigos nos seus tempos livres, praticar exercício, principalmente perto da natureza.
    \item \textbf{Contexto de uso da aplicação:} Paulo Monteiro faz deslocações diárias para Universidade sozinho no seu carro, e estaria disponível para partilhar o carro para reduzir as suas despesas, e ao mesmo tempo contribuir para o meio ambiente. Para além das viagens diárias para a Universidade, Paulo vai frequentemente a Lisboa, para passar tempo com a sua namorada, que se encontra a estudar nessa mesma cidade, sendo estas viagens muito dispendiosas. Nas viagens a Lisboa, Paulo procura sempre por pessoas, que façam essa mesma viagem, para tentar dividir as despesas, mas torna-se difícil.
\end{itemize}{}

\subsection{Bruna Pereira}
\begin{itemize}
    \item Idade: 20 anos
    \item Estado Civil: Solteira
    \item Habilitações: 12º Ano
    \item Profissão: Estudante
    \item Residência: Braga
    \item \textbf{Life style:} A Bruna Pereira gosta de sair com os amigos nos seus tempos livres, praticar exercício, principalmente perto da natureza, viajar, ler.
    \item \textbf{Contexto de uso da aplicação:} Bruna, estuda em Lisboa, e regressa a Braga, muito frequentemente (aproximadamente, de quinze em quinze dias). Sendo a viagem dispendiosa, principalmente para um estudante, Bruna tenta sempre procurar outros estudantes de Braga, Porto ou Viana do Castelo, para dividirem o carro, não sendo fácil por vezes encontrar pessoas para tal. A Bruna preocupa-se muito com o estado ambiental, tentando sempre contribuir para o mesmo.
\end{itemize}{}

\subsection{Pedro Lima}
\begin{itemize}
    \item Idade: 30 anos;
    \item Estado Civil: Casado
    \item Habilitações: Licenciado em Engenharia Informática
    \item Profissão: Engenheiro Informático na Microsoft Porto
    \item Residência: Braga
    \item \textbf{Life style:} O Pedro Lima gosta de praticar exercício, principalmente perto da natureza, viajar, ler. 
    \item \textbf{Contexto de uso da aplicação:} Pedro mora em Braga e trabalha no Porto. Ele faz a viagem de ida e volta entre Braga e Porto todos os dias da semana.
    
\end{itemize}{}

\section{Atribuição de Prioridades a Utilizadores}

\subsection{Utilizadores chaves}
\begin{itemize}
    \item Utilizadores que usam a aplicação frequentemente, deslocações diárias.
    \item Estudantes universitários.
    \item Trabalhadores de empresas.
\end{itemize}{}
\subsection{Utilizadores secundários}
\begin{itemize}
    \item Utilizadores que usam a aplicação para deslocações pouco frequentes.
\end{itemize}{}
\subsection{Utilizadores sem importância}
\begin{itemize}
    \item Todos os utilizadores que tenham como objetivo, causar danos a outros utilizadores, como por exemplo assaltos, etc ...
\end{itemize}{}

\section{Participação dos Utilizadores}

\hspace{5mm} A participação dos utilizadores, para uma melhor perceção das funcionalidades importantes, torna-se necessário visto que o produto vai ser desenvolvido para os mesmos. 

\hspace{5mm} Deste modo, espera-se que os possíveis utilizadores, com as mesmas características das personas, possam contribuir com respostas a inquéritos e entrevistas. Na verdade, a nível de tempo, que se espera da participação dos utilizadores, consiste no tempo para responder aos inquéritos ou entrevistas, ou seja, de cinco a dez minutos.  

\section{Manutenção do Produto com a Participação dos Utilizadores}
\hspace{5mm} Com o objetivo de se manter um produto atualizado com as preferências dos utilizadores, serão feitos inquéritos e entrevistas, após o lançamento do produto, para se perceber, o que pode estar mal na aplicação, ou o que falta.

\hspace{5mm} No geral, o objetivo consiste em resolver possíveis problemas encontrados pelos utilizadores e solucionar novas funcionalidades em falta.
