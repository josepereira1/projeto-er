\chapter{Limitações}
\hspace{5mm} Nesta capítulo, descreve-se as limitações mais relevantes em torno do produto. As limitações podem estar relacionadas ao sistema em si, aos \emph{stakeholders}, ao mercado, ao ambiente, etc.

\hspace{5mm} Algumas dessas limitações podem ser temporárias e transponíveis ao longo do tempo. Por outro lado, outras limitações são definitivas, e o sistema deve funcionar apesar delas.

\section{Limitações da Solução}\label{0:0.3.1}

\begin{table}[!h]
\begin{center}
\begin{tabularx}{\textwidth}{ | c | X | }
\hline
  Id da Limitação & 1  \\
  \hline
  Descrição & O produto requer um telemóvel com conexão à Internet. \\
  \hline
  Justificação & Para que os consumidores consigam ver a lista de deslocações disponíveis, solicitar uma vaga (passageiro) e aceitar a solicitação (motorista), é necessário um  telemóvel com acesso à Internet. Além disso, para iniciar e finalizar a deslocação, bem como avaliar os participantes, também é necessário esse acesso. \\
  \hline
\end{tabularx}
\caption{Limitação 1.} \label{tab:r1}
\end{center}
\end{table}

\begin{table}[!h]
\begin{center}
    \begin{tabularx}{\textwidth}{ | c | X | }
        \hline
        Id da Limitação & 2 \\
        \hline
        Descrição & O produto requer um telemóvel com GPS. \\
        \hline
        Justificação & Para que o utilizador consiga acompanhar o deslocamento, é necessário que seu telemóvel possua GPS. \\
        \hline
    \end{tabularx}
    \caption{Limitação 2.} \label{tab:r2}
\end{center}
\end{table}

\begin{table}[!h]
\begin{center}
    \begin{tabularx}{\textwidth}{ | c | X | }
    \hline
    Id da Limitação & 3 \\
    \hline
    Descrição & O produto deve funcionar tanto em dispositivos Android quanto Apple. \\
    \hline
    Justificação & Para poder abranger uma parcela grande da população, é necessário que a aplicação possa ser utilizada tanto nos dispositivos da Apple quanto nos Android. \\
    \hline
    \end{tabularx}
    \caption{Limitação 3} \label{tab:r3}
\end{center}
\end{table}

\section{Ambiente da Implementação do Sistema}\label{0:0.3.2}

\begin{table}[!h]
\begin{center}
    \begin{tabularx}{\textwidth}{ | c | X | }
    \hline
    Id da Limitação & 4 \\
    \hline
    Descrição & O sistema deverá rodar em um ambiente Cloud. \\
    \hline
    Justificação & Espera-se que o sistema seja utilizado por uma grande quantidade de utilizadores, e que faça o processamento de imensas quantidades de dados. Para que isso seja possível, será necessário instalá-lo em um ambiente Cloud, com alta capacidade de escalabilidade e disponibilidade. \\
    \hline
    \end{tabularx}
    \caption{Limitação 4} \label{tab:r4}
\end{center}
\end{table}

\section{Aplicações Parceiras ou Colaborativas}\label{0:0.3.3}

\begin{table}[!h]
\begin{center}
    \begin{tabularx}{\textwidth}{ | c | X | }
    \hline
    Id da Limitação & 4 \\
    \hline
    Descrição & O produto terá como colaboração a aplicação da Michelin. \\
    \hline
    Justificação & Para o cálculo dos custos das deslocações. \\
    \hline
    \end{tabularx}
    \caption{Limitação 3} \label{tab:r3}
\end{center}
\end{table}

\section{Software de Prateleira}\label{0:0.3.4}

\begin{table}[!h]
\begin{center}
    \begin{tabularx}{\textwidth}{ | c | X | }
    \hline
    Id da Limitação & 5 \\
    \hline
    Descrição & O produto precisa integrar com aplicações de Mapas. \\
    \hline
    Justificação & Para que os utilizadores consigam visualizar as rotas, antes e durante o deslocamento, é necessário que o produto utilize uma solução de mapas integrado (como Google Maps, por exemplo). \\
    \hline
    \end{tabularx}
    \caption{Limitação 5} \label{tab:r5}
\end{center}
\end{table}

\section{Ambiente de Uso do Sistema}\label{0:0.3.5}

\begin{table}[!h]
\begin{center}
    \begin{tabularx}{\textwidth}{ | c | X | }
    \hline
    Id da Limitação & 6 \\
    \hline
    Descrição & O produto deve emitir instruções sonoras. \\
    \hline
    Justificação & Como um dos utilizadores estará a conduzir o veículo, é importante que o produto emita instruções sonoras, para evitar que o mesmo precise manusear o telemóvel enquando conduz. \\
    \hline
    \end{tabularx}
    \caption{Limitação 6} \label{tab:r6}
\end{center}
\end{table}

\begin{table}[!h]
\begin{center}
    \begin{tabularx}{\textwidth}{ | c | X | }
    \hline
    Id da Limitação & 7 \\
    \hline
    Descrição & O produto deve ter bom contraste e brilho. \\
    \hline
    Justificação & Os utilizadores irão interagir com o produto, na maior parte, em ambientes externos. Dessa forma, é importante que a aplicação tenha bom contraste e brilho, a fim de facilitar seu uso em ambientes de forte luminosidade (como em dias de Sol, por exemplo). \\
    \hline
    \end{tabularx}
    \caption{Limitação 7} \label{tab:r7}
\end{center}
\end{table}

\section{Limitações da Empresa}\label{0:0.3.8}

\begin{table}[!h]
\begin{center}
    \begin{tabularx}{\textwidth}{ | c | X | }
    \hline
    Id da Limitação & 8 \\
    \hline
    Descrição & A empresa deve tratar os dados pessoais de acordo com o RGPD. \\
    \hline
    Justificação & A fim de inspirar confiança ao utilizador, bem como evitar eventuais multas, a empresa precisa obter, processar, armezenar e remover todos os dados pessoais seguindo práticas seguras, e que garantam a aderência às exigências do RGPD. \\
    \hline
    \end{tabularx}
    \caption{Limitação 8} \label{tab:r8}
\end{center}
\end{table}