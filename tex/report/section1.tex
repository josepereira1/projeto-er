\chapter{O propósito do projeto}

\section{Contextualização}

Cada vez mais percebe-se um aumento na quantidade de veículos a circular nas ruas das cidades - especialmente as grandes cidades. Esse alto volume de automóveis gera diversos problemas, tanto para a sociedade quanto para o meio ambiente. \newline O tempo gasto pelas pessoas nos seus deslocamentos diários tem sido cada vez maior, o que diminui muito a sua qualidade de vida. Além disso, questões como poluição sonora e poluição do ar colocam em risco a saúde das pessoas e do planeta como um todo. \newline Deste modo, a nossa empresa X analisou um dos grandes promotores deste problema: a não utilização inteligente dos recursos automóveis. \newline A má utilização refere-se ao facto de maior parte dos automóveis que circulam nas estradas conterem muitos lugares, no entanto, apenas utilizados por poucas ou uma única pessoa.\newline Segundo um estudo nos Estados Unidos da América, as pessoas passam 75\% do tempo a conduzir sozinhas o seu carro.

\section{Objetivos do Projeto}

Deste modo, a solução para este problema, foi a criação de uma aplicação que reúne pessoas que façam as mesmas deslocações, nos mesmos horários, e assim poderem partilhar um automóvel, dividirem as despesas e ao mesmo tempo contribuir para o meio ambiente.\newline \newline Portanto, imagine-se um exemplo: A Maria tem um carro de cinco lugares, regista-se na aplicação e disponibiliza uma deslocação que irá fazer no dia 19 de Dezembro de 2019, de Braga até Lisboa, às \emph{9h00} e regressa às \emph{20h00} do mesmo dia. \newline O João e a Rita, são de Braga, pretendem ir a Lisboa nos mesmos horários que a Maria. Como estão todos  registados, a aplicação vai reuni-los para a partilha do automóvel, dividindo as despesas da deslocação. \newline Pontos positivos deste exemplo, evitou-se a má utilização de três veículos (Maria, João e Rita), reduz-se os preços da deslocação para cada um, e contribui-se para a redução da poluição para o meio ambiente e a quantidade de automóveis a circular pelas estradas. Pensando neste exemplo a larga escala, ou seja, milhares de utilizadores, a pouco e pouco, vai-se aproveitar melhor os recursos automóveis e por conseguinte reduzir as emissões prejudiciais ao ambiente.