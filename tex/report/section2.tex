\chapter{Partes Interessadas - The Stakeholders}
\section{Cliente \emph{in english the Client}}\label{0:0.2.1}
O \emph{cliente} deste projeto é a administração da empresa \textbf{X}, também responsável pelo desenvolvimento do mesmo.

\section{Consumidor \emph{in english the Customer}}\label{0:0.2.2}
O consumidor da aplicação \textbf{ShareCar}, serão todas as pessoas que deslocam-se entre dois ou mais pontos, onde seja necessária a utilização de um automóvel e que estejam disponíveis para partilhar com outras pessoas. Exemplos de pessoas para consumidores serão: universitários, pessoas que trabalham no mesmo lugar ou próximo, etc.

\section{Consumidores de outros sistemas}\label{0:0.2.3}

Uma vez já existem no mercado atual sistemas que têm um objetivo semelhante ao da nossa aplicação é importante analisar a concorrência e contabilizar os consumidores desses sistemas como possíveis partes interessadas.
\newline
Assim identificamos três sistemas a ter em conta: \href{https://www.uber.com}{Uber}, \href{https://bolt.eu/en/}{Bolt} e \href{https://www.blablacar.pt/}{BlaBlaCar}. 


\section{Outras partes interessadas}
\begin{itemize}
    \item Cliente (\ref{0:0.2.1})
    \item Consumidor (\ref{0:0.2.2})
    \item Consumidores de outros sistemas (\ref{0:0.2.3})
    \item Ministério do Ambiente
    \item Ministério das Finanças
    \item Orgãos regulamentadores
    \item Colaboradores da empresa X
    \item Investidores da empresa X
    \item Ambientalistas
    \item Organização não governamentais
\end{itemize}

\section{Utilizadores do Produto}
Os utilizadores do produto são aqueles interessados em utilizar a aplicação para encontrar pessoas, a fim de dividir despesas e partilhar um automóvel ao se deslocarem entre dois ou mais pontos. Destes utilizadores, pode-se destacar os seguintes:

\begin{itemize}
        \item Professores de Universidades, escolas Básicas e Secundárias
        \begin{itemize}
            \item Professores de uma mesma escola que residam proximamente podem partilhar um automóvel.
            \item O papel deste utilizador será de motorista ou de passageiro.
            \item Os conhecimentos tecnológicos necessários serão a capacidade de abrir e utilizar a aplicação \textbf{ShareCar} em um smartphone, Desktop ou Laptop.
        \end{itemize}{}
        \item Estudantes Universitários
        \begin{itemize}
            \item Estudantes da mesma universidade ou de universidades locais e que residam proximamente, podem partilhar o automóvel.
            \item O papel deste utilizador será de motorista ou de passageiro.
            \item Os conhecimentos tecnológicos necessários serão a capacidade de abrir e utilizar a aplicação \textbf{ShareCar} em um smartphone, Desktop ou Laptop.
        \end{itemize}{}
        \item Pessoas da mesma zona residêncial que trabalham no mesmo lugar ou próximo
        \begin{itemize}
            \item O papel deste utilizador será de motorista ou de passageiro.
            \item Os conhecimentos tecnológicos necessários serão a capacidade de abrir e utilizar a aplicação \textbf{ShareCar} em um smartphone, Desktop ou Laptop.
        \end{itemize}{}
        \item Pessoas que costumam fazer grandes viagens frequentemente (p.e. Braga - Lisboa)\begin{itemize}
            \item O papel deste utilizador será de motorista ou de passageiro.
            \item Os conhecimentos tecnológicos necessários serão a capacidade de abrir e utilizar a aplicação \textbf{ShareCar} em um smartphone, Desktop ou Laptop.
        \end{itemize}{}
\end{itemize}{}

\section{Personas}
De seguida apresentam-se três Personas possíveis da aplicação \textbf{ShareCar}.

\subsection{Persona Paulo Monteiro}

\begin{itemize}
    \item Nome: Paulo Monteiro;
    \item Idade: 25 anos;
    \item Trabalho: Engenheiro Informático na Accenture;
    \item Tempos livres: Joga Futebol;
    \item Vive: Vila Verde;
    \item Comida favorita: Arroz de pato;
    \item Música favorita: Bárbara Tinoco - Antes de dizer que sim;
    \item Gosta: Natureza, passear, viajar, comer, sair com amigos;
    \item Não Gosta: Arroz de sangue;
    \item Aos fins de semana: Gosta de sair com amigos e ir ter com a namorada a Lisboa;
    \item Atitude sobre tecnologia: Gosta novas aplicações;
    \item Atitude com dinheiro: Muito poupado.
\end{itemize}{}

\subsection{Persona Bruna Pereira}

\begin{itemize}
    \item Nome: Bruna Pereira;
    \item Idade: 21 anos;
    \item Trabalho: Estudante na Universidade do Minho;
    \item Tempos livres: karaté;
    \item Vive: Vila Verde;
    \item Comida favorita: Bacalhau à Brás;
    \item Música favorita: Billie Eilish - I love you;
    \item Gosta: Natureza, viajar, sair com amigos;
    \item Não Gosta: Futebol;
    \item Aos fins de semana: Gosta de sair com amigos e ir ter com o namorado a Lisboa;
    \item Atitude sobre tecnologia: Gosta novas aplicações;
    \item Atitude com dinheiro: Muito poupada.
\end{itemize}{}

\subsection{Persona Pedro Lima}

\begin{itemize}
    \item Nome: Pedro Lima;
    \item Idade: 30 anos;
    \item Trabalho: Profesor na Universidade do Minho;
    \item Tempos livres: Basketball;
    \item Vive: Vila Verde;
    \item Comida favorita: Bacalhau com natas;
    \item Música favorita: Alen Walker - Sky;
    \item Gosta: viajar, sair com amigos;
    \item Não Gosta: Milho;
    \item Aos fins de semana: Gosta de sair com amigos;
    \item Atitude sobre tecnologia: Gosta novas aplicações;
    \item Atitude com dinheiro: Muito poupado.
\end{itemize}{}

\section{Atribuição de Prioridades a Utilizadores}

\subsection{Utilizadores chaves}
\begin{itemize}
    \item Utilizadores que usam a aplicação frequentemente, deslocações diárias.
    \item Estudantes universitários.
    \item Trabalhadores de empresas.
\end{itemize}{}
\subsection{Utilizadores secundários}
\begin{itemize}
    \item Utilizadores que usam a aplicação para deslocações pouco frequentes.
\end{itemize}{}
\subsection{Utilizadores sem importância}
\begin{itemize}
    \item Todos os utilizadores que tenham como objetivo, causar danos a outros utilizadores, como por exemplo assaltos, etc ...
\end{itemize}{}

\section{User Participation}

\section{Maintenance Users and Service Technicians}
